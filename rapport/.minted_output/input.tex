%%
% Copyright (c) 2017 - 2021, Pascal Wagler;
% Copyright (c) 2014 - 2021, John MacFarlane
%
% All rights reserved.
%
% Redistribution and use in source and binary forms, with or without
% modification, are permitted provided that the following conditions
% are met:
%
% - Redistributions of source code must retain the above copyright
% notice, this list of conditions and the following disclaimer.
%
% - Redistributions in binary form must reproduce the above copyright
% notice, this list of conditions and the following disclaimer in the
% documentation and/or other materials provided with the distribution.
%
% - Neither the name of John MacFarlane nor the names of other
% contributors may be used to endorse or promote products derived
% from this software without specific prior written permission.
%
% THIS SOFTWARE IS PROVIDED BY THE COPYRIGHT HOLDERS AND CONTRIBUTORS
% "AS IS" AND ANY EXPRESS OR IMPLIED WARRANTIES, INCLUDING, BUT NOT
% LIMITED TO, THE IMPLIED WARRANTIES OF MERCHANTABILITY AND FITNESS FOR A PARTICULAR PURPOSE ARE DISCLAIMED. IN NO EVENT SHALL THE
% COPYRIGHT OWNER OR CONTRIBUTORS BE LIABLE FOR ANY DIRECT, INDIRECT,
% INCIDENTAL, SPECIAL, EXEMPLARY, OR CONSEQUENTIAL DAMAGES (INCLUDING,
% BUT NOT LIMITED TO, PROCUREMENT OF SUBSTITUTE GOODS OR SERVICES;
% LOSS OF USE, DATA, OR PROFITS; OR BUSINESS INTERRUPTION) HOWEVER
% CAUSED AND ON ANY THEORY OF LIABILITY, WHETHER IN CONTRACT, STRICT
% LIABILITY, OR TORT (INCLUDING NEGLIGENCE OR OTHERWISE) ARISING IN
% ANY WAY OUT OF THE USE OF THIS SOFTWARE, EVEN IF ADVISED OF THE
% POSSIBILITY OF SUCH DAMAGE.
%%

%%
% This is the Eisvogel pandoc LaTeX template.
%
% For usage information and examples visit the official GitHub page:
% https://github.com/Wandmalfarbe/pandoc-latex-template
%%

% Options for packages loaded elsewhere
\PassOptionsToPackage{unicode}{hyperref}
\PassOptionsToPackage{hyphens}{url}
\PassOptionsToPackage{dvipsnames,svgnames*,x11names*,table}{xcolor}
%
\documentclass[
  11pt,
  paper=a4,
  ,captions=tableheading
]{scrartcl}
\usepackage{amsmath,amssymb}
\usepackage{lmodern}
\usepackage{setspace}
\setstretch{1.2}
\usepackage{ifxetex,ifluatex}
\ifnum 0\ifxetex 1\fi\ifluatex 1\fi=0 % if pdftex
  \usepackage[T1]{fontenc}
  \usepackage[utf8]{inputenc}
  \usepackage{textcomp} % provide euro and other symbols
\else % if luatex or xetex
  \usepackage{unicode-math}
  \defaultfontfeatures{Scale=MatchLowercase}
  \defaultfontfeatures[\rmfamily]{Ligatures=TeX,Scale=1}
\fi
% Use upquote if available, for straight quotes in verbatim environments
\IfFileExists{upquote.sty}{\usepackage{upquote}}{}
\IfFileExists{microtype.sty}{% use microtype if available
  \usepackage[]{microtype}
  \UseMicrotypeSet[protrusion]{basicmath} % disable protrusion for tt fonts
}{}
\makeatletter
\@ifundefined{KOMAClassName}{% if non-KOMA class
  \IfFileExists{parskip.sty}{%
    \usepackage{parskip}
  }{% else
    \setlength{\parindent}{0pt}
    \setlength{\parskip}{6pt plus 2pt minus 1pt}}
}{% if KOMA class
  \KOMAoptions{parskip=half}}
\makeatother
\usepackage{xcolor}
\definecolor{default-linkcolor}{HTML}{A50000}
\definecolor{default-filecolor}{HTML}{A50000}
\definecolor{default-citecolor}{HTML}{4077C0}
\definecolor{default-urlcolor}{HTML}{4077C0}

\definecolor{sol-yellow}{HTML}{B58900}
\definecolor{sol-orange}{HTML}{CB4B16}
\definecolor{sol-red}{HTML}{DC322F}
\definecolor{sol-magenta}{HTML}{D33682}
\definecolor{sol-violet}{HTML}{6C71C4}
\definecolor{sol-blue}{HTML}{268BD2}
\definecolor{sol-cyan}{HTML}{2AA198}
\definecolor{sol-green}{HTML}{859900}

\IfFileExists{xurl.sty}{\usepackage{xurl}}{} % add URL line breaks if available
\IfFileExists{bookmark.sty}{\usepackage{bookmark}}{\usepackage{hyperref}}
\hypersetup{
  pdftitle={Rapport projet Sécurité des applications},
  pdfauthor={Sermeus Steven; Frippiat Gabriel},
  hidelinks,
  breaklinks=true,
  pdfcreator={LaTeX via pandoc with the Eisvogel template}}
\urlstyle{same} % disable monospaced font for URLs
\usepackage[margin=2.5cm,includehead=true,includefoot=true,centering,]{geometry}

  \usepackage{fontawesome5}
  \usepackage[outputdir=.minted_output]{minted}
  \definecolor{solbg}{HTML}{efece2}
  \setmintedinline{bgcolor={}}
  \makeatletter
  \let\listoflistings\@undefined
  \makeatother
    \setminted{fontsize=\small}
  


\usepackage{longtable,booktabs,array}
\usepackage{calc} % for calculating minipage widths
% Correct order of tables after \paragraph or \subparagraph
\usepackage{etoolbox}
\makeatletter
\patchcmd\longtable{\par}{\if@noskipsec\mbox{}\fi\par}{}{}
\makeatother
% Allow footnotes in longtable head/foot
\IfFileExists{footnotehyper.sty}{\usepackage{footnotehyper}}{\usepackage{footnote}}
\makesavenoteenv{longtable}
% add backlinks to footnote references, cf. https://tex.stackexchange.com/questions/302266/make-footnote-clickable-both-ways
\usepackage{footnotebackref}
\usepackage{graphicx}
\makeatletter
\def\maxwidth{\ifdim\Gin@nat@width>\linewidth\linewidth\else\Gin@nat@width\fi}
\def\maxheight{\ifdim\Gin@nat@height>\textheight\textheight\else\Gin@nat@height\fi}
\makeatother
% Scale images if necessary, so that they will not overflow the page
% margins by default, and it is still possible to overwrite the defaults
% using explicit options in \includegraphics[width, height, ...]{}
\setkeys{Gin}{width=\maxwidth,height=\maxheight,keepaspectratio}
% Set default figure placement to htbp
\makeatletter
\def\fps@figure{htbp}
\makeatother
\setlength{\emergencystretch}{3em} % prevent overfull lines
\providecommand{\tightlist}{%
  \setlength{\itemsep}{0pt}\setlength{\parskip}{0pt}}
\setcounter{secnumdepth}{5}

% Make use of float-package and set default placement for figures to H.
% The option H means 'PUT IT HERE' (as  opposed to the standard h option which means 'You may put it here if you like').
\usepackage{float}
\floatplacement{figure}{H}



  \usepackage[french]{babel}

\usepackage{fontawesome5}
\usepackage[outputdir=.minted_output]{minted}
\definecolor{solbg}{HTML}{efece2}
\setmintedinline{bgcolor={}}
\usepackage{tikz}
\usetikzlibrary{arrows,calc,shapes,automata,backgrounds,petri,fit,mindmap,decorations.pathmorphing,patterns,intersections,trees,positioning}
\makeatletter
\let\listoflistings\@undefined
\makeatother
\makeatletter
\@ifpackageloaded{subfig}{}{\usepackage{subfig}}
\@ifpackageloaded{caption}{}{\usepackage{caption}}
\captionsetup[subfloat]{margin=0.5em}
\AtBeginDocument{%
\renewcommand*\figurename{Figure}
\renewcommand*\tablename{Table}
}
\AtBeginDocument{%
\renewcommand*\listfigurename{List of Figures}
\renewcommand*\listtablename{List of Tables}
}
\newcounter{pandoccrossref@subfigures@footnote@counter}
\newenvironment{pandoccrossrefsubfigures}{%
\setcounter{pandoccrossref@subfigures@footnote@counter}{0}
\begin{figure}\centering%
\gdef\global@pandoccrossref@subfigures@footnotes{}%
\DeclareRobustCommand{\footnote}[1]{\footnotemark%
\stepcounter{pandoccrossref@subfigures@footnote@counter}%
\ifx\global@pandoccrossref@subfigures@footnotes\empty%
\gdef\global@pandoccrossref@subfigures@footnotes{{##1}}%
\else%
\g@addto@macro\global@pandoccrossref@subfigures@footnotes{, {##1}}%
\fi}}%
{\end{figure}%
\addtocounter{footnote}{-\value{pandoccrossref@subfigures@footnote@counter}}
\@for\f:=\global@pandoccrossref@subfigures@footnotes\do{\stepcounter{footnote}\footnotetext{\f}}%
\gdef\global@pandoccrossref@subfigures@footnotes{}}
\@ifpackageloaded{float}{}{\usepackage{float}}
\floatstyle{ruled}
\@ifundefined{c@chapter}{\newfloat{codelisting}{h}{lop}}{\newfloat{codelisting}{h}{lop}[chapter]}
\floatname{codelisting}{Listing}
\newcommand*\listoflistings{\listof{codelisting}{List of Listings}}
\makeatother
  \usepackage{polyglossia}
  \setmainlanguage[]{french}
\ifluatex
  \usepackage{selnolig}  % disable illegal ligatures
\fi

\title{Rapport projet Sécurité des applications}
\author{Sermeus Steven \and Frippiat Gabriel}
\date{}



%%
%% added
%%

%
% language specification
%
% If no language is specified, use English as the default main document language.
%

\ifnum 0\ifxetex 1\fi\ifluatex 1\fi=0 % if pdftex
  \usepackage[shorthands=off,main=english]{babel}
\else
    % Workaround for bug in Polyglossia that breaks `\familydefault` when `\setmainlanguage` is used.
  % See https://github.com/Wandmalfarbe/pandoc-latex-template/issues/8
  % See https://github.com/reutenauer/polyglossia/issues/186
  % See https://github.com/reutenauer/polyglossia/issues/127
  \renewcommand*\familydefault{\sfdefault}
    % load polyglossia as late as possible as it *could* call bidi if RTL lang (e.g. Hebrew or Arabic)
  \usepackage{polyglossia}
  \setmainlanguage[]{english}
\fi



%
% for the background color of the title page
%
\usepackage{pagecolor}
\usepackage{afterpage}
\usepackage[margin=2.5cm,includehead=true,includefoot=true,centering]{geometry}

%
% break urls
%
\PassOptionsToPackage{hyphens}{url}

%
% When using babel or polyglossia with biblatex, loading csquotes is recommended
% to ensure that quoted texts are typeset according to the rules of your main language.
%
\usepackage{csquotes}

%
% captions
%
\definecolor{caption-color}{HTML}{777777}
\usepackage[font={stretch=1.2}, textfont={color=caption-color}, position=top, skip=4mm, labelfont=bf, singlelinecheck=false, justification=raggedright]{caption}
\setcapindent{0em}

%
% blockquote
%
\definecolor{blockquote-border}{RGB}{221,221,221}
\definecolor{blockquote-text}{RGB}{119,119,119}
\usepackage{mdframed}
\newmdenv[rightline=false,bottomline=false,topline=false,linewidth=3pt,linecolor=blockquote-border,skipabove=\parskip]{customblockquote}
\renewenvironment{quote}{\begin{customblockquote}\list{}{\rightmargin=0em\leftmargin=0em}%
\item\relax\color{blockquote-text}\ignorespaces}{\unskip\unskip\endlist\end{customblockquote}}

%
% Source Sans Pro as the de­fault font fam­ily
% Source Code Pro for monospace text
%
% 'default' option sets the default
% font family to Source Sans Pro, not \sfdefault.
%
\ifnum 0\ifxetex 1\fi\ifluatex 1\fi=0 % if pdftex
    \usepackage[default]{sourcesanspro}
  \usepackage{sourcecodepro}
  \else % if not pdftex
    \usepackage[default]{sourcesanspro}
  \usepackage{sourcecodepro}

  % XeLaTeX specific adjustments for straight quotes: https://tex.stackexchange.com/a/354887
  % This issue is already fixed (see https://github.com/silkeh/latex-sourcecodepro/pull/5) but the
  % fix is still unreleased.
  % TODO: Remove this workaround when the new version of sourcecodepro is released on CTAN.
  \ifxetex
    \makeatletter
    \defaultfontfeatures[\ttfamily]
      { Numbers   = \sourcecodepro@figurestyle,
        Scale     = \SourceCodePro@scale,
        Extension = .otf }
    \setmonofont
      [ UprightFont    = *-\sourcecodepro@regstyle,
        ItalicFont     = *-\sourcecodepro@regstyle It,
        BoldFont       = *-\sourcecodepro@boldstyle,
        BoldItalicFont = *-\sourcecodepro@boldstyle It ]
      {SourceCodePro}
    \makeatother
  \fi
  \fi

%
% heading color
%
\definecolor{heading-color}{RGB}{40,40,40}
\addtokomafont{section}{\color{heading-color}}
% When using the classes report, scrreprt, book,
% scrbook or memoir, uncomment the following line.
%\addtokomafont{chapter}{\color{heading-color}}

%
% variables for title, author and date
%
\usepackage{titling}
\title{Rapport projet Sécurité des applications}
\author{Sermeus Steven, Frippiat Gabriel}
\date{}

%
% tables
%

\definecolor{table-row-color}{HTML}{F5F5F5}
\definecolor{table-rule-color}{HTML}{999999}

%\arrayrulecolor{black!40}
\arrayrulecolor{table-rule-color}     % color of \toprule, \midrule, \bottomrule
\setlength\heavyrulewidth{0.3ex}      % thickness of \toprule, \bottomrule
\renewcommand{\arraystretch}{1.3}     % spacing (padding)


%
% remove paragraph indention
%
\setlength{\parindent}{0pt}
\setlength{\parskip}{6pt plus 2pt minus 1pt}
\setlength{\emergencystretch}{3em}  % prevent overfull lines

%
%
% Listings
%
%


\usepackage[multiple]{footmisc}

%
% header and footer
%
\usepackage{fancyhdr}

\fancypagestyle{eisvogel-header-footer}{
  \fancyhead{}
  \fancyfoot{}
  \lhead[]{Rapport projet Sécurité des applications}
  \chead[]{}
  \rhead[Rapport projet Sécurité des applications]{}
  \lfoot[\thepage]{Sermeus Steven, Frippiat Gabriel}
  \cfoot[]{}
  \rfoot[Sermeus Steven, Frippiat Gabriel]{\thepage}
  \renewcommand{\headrulewidth}{0.4pt}
  \renewcommand{\footrulewidth}{0.4pt}
}
\pagestyle{eisvogel-header-footer}

%%
%% end added
%%



\begin{document}

%%
%% begin titlepage
%%
\begin{titlepage}
\newgeometry{left=6cm}
\newcommand{\colorRule}[3][black]{\textcolor[HTML]{#1}{\rule{#2}{#3}}}
\begin{flushleft}
\noindent
\\[-1em]
\color[HTML]{5F5F5F}
\makebox[0pt][l]{\colorRule[435488]{1.3\textwidth}{4pt}}
\par
\noindent

{
  \setstretch{1.4}
  \vfill
  \noindent {\huge \textbf{\textsf{Rapport projet Sécurité des
applications}}}
    \vskip 2em
  \noindent {\Large \textsf{Sermeus Steven, Frippiat Gabriel}}
  \vfill
}


\textsf{}
\end{flushleft}
\end{titlepage}
\restoregeometry

%%
%% end titlepage
%%



\newpage

\renewcommand{\contentsname}{Table des matières}

\tableofcontents
\newpage

\hypertarget{introduction}{%
\section{Introduction}\label{introduction}}

Pour ce projet, nous avions comme objectif de développer une application
permettant de partager des photos avec du chiffrement de bout en bout.
L’application avait comme contrainte d’utiliser l’architecture
client-serveur où le serveur ne peut pas être considéré comme une entité
de confiance. Ce serveur permet d’enregistrer de nouveaux utilisateurs,
d’authentifier des utilisateurs et d’autoriser des utilisateurs
authentifiés d’effectuer des actions sur l’application.

Les fonctionnalités attendues étaient :

\begin{itemize}
\tightlist
\item
  La création d’un album photos
\item
  L’upload de photos
\item
  Le partage de photos
\item
  Le partage d’album
\item
  La création d’un compte
\item
  S’authentifier sur l’application
\item
  Pouvoir utiliser l’application sur plusieurs devices
\end{itemize}

Pour réaliser ce projet, nous avons utilisé différentes technologies qui
sont présentées et détaillées dans la suite de ce rapport.

\newpage

\hypertarget{architecture-physique}{%
\section{Architecture physique}\label{architecture-physique}}

Pour répondre à la problématique de l’énoncé tout en respectant les
contraintes, nous avons utilisé une architecture client-serveur où le
serveur ne peut pas être considéré comme un acteur de confiance. De ce
fait, il a été nécessaire de mettre en place un chiffrement de bout en
bout des informations sensibles. De plus, la sécurité des différents
composants de l’application devait être assurée. Pour cela, nous avons
utilisé différentes technologies pour chaque composant de
l’architecture.

\begin{figure}
\centering
\includegraphics{/tmp/tex2pdf.-9228386339f5691c/af3941e5feca7e69e61e8e1f3455d6518b7164e5.png}
\caption{Architecture physique}
\end{figure}

Comme le montre le schéma ci-dessus, l’architecture physique de notre
application est composée de plusieurs éléments :

\begin{itemize}
\item
  \textbf{Base de données} : Celle-ci permet de stocker les informations
  des utilisateurs à long terme.
\item
  \textbf{Reverse proxy} : Il permet d’assurer la connexion HTTPS entre
  le client et l’infrastructure. Nous avons utilisé Nginx, il assure
  également la connexion en HTTPS entre le serveur et le client.
\item
  \textbf{Serveur de fichiers} : L’utilisateur pouvant être malveillant,
  nous avons décidé de ne pas stocker les fichiers sur le serveur
  principal. Nous avons délégué cette tâche à un serveur de fichiers
  dédié, Minio.
\item
  \textbf{Serveur de logs} : Pour avoir un stockage de log fiable et
  sécurisé, nous avons utilisé un serveur de logs dédié, Seq.
\item
  \textbf{Base de données clés valeurs} : Celle-ci permet d’effectuer
  des opérations de lecture et d’écriture rapides et nous a été utile
  pour effectuer du rate limiting sur les différentes routes de
  l’application. Nous avons utilisé Valkey, qui est le fork open-source
  de Redis.
\item
  \textbf{Next.js} : Il s’agit du coeur de l’application, il permet de
  gérer les différentes routes et de générer les pages côté serveur.
\end{itemize}

\hypertarget{cryptographie}{%
\section{Cryptographie}\label{cryptographie}}

Dans le cadre de ce projet, nous avons utilisé deux types de
cryptographie : la cryptographie symétrique et la cryptographie
asymétrique.

\hypertarget{cryptographie-asymuxe9trique}{%
\subsection{Cryptographie
asymétrique}\label{cryptographie-asymuxe9trique}}

Celle-ci est utilisée pour l’authentification des utilisateurs ainsi que
le partage de clés de chiffrement symétrique. Nous avons utilisé la
librairie standard de JavaScript, \mintinline[]{c}{crypto}, pour générer
des clés RSA et les utiliser pour chiffrer et déchiffrer des données. La
librairie standard permet uniquement d’utiliser l’algorithme RSA pour de
chiffrement asymétrique. Nous avons utilisé une clé de taille 4096, même
si cette taille de clé est suspectée d’être “cassable”. Pour le moment
elle reste sécurisée et, en cas de besoin, il est toujours possible
d’augmenter la taille de la clé.

\hypertarget{authentification}{%
\subsubsection{Authentification}\label{authentification}}

Lorsqu’un utilisateur s’inscrit sur notre application, un couple de clés
RSA est généré. La clé publique de l’utilisateur est stockée dans la
base de données et la clé privée est stockée dans le navigateur de
l’utilisateur. Si l’utilisateur souhaite se connecter, il doit
déchiffrer un challenge qui lui est envoyé par le serveur. Ce challenge
est chiffré avec la clé publique de l’utilisateur. Si l’utilisateur
parvient à déchiffrer le challenge, il est considéré comme authentifié
étant donné que seul l’utilisateur possède la clé privée correspondante
à la clé publique stockée dans la base de données permettant de
déchiffrer le challenge.

\hypertarget{partage-de-cluxe9s-de-chiffrement-symuxe9trique}{%
\subsubsection{Partage de clés de chiffrement
symétrique}\label{partage-de-cluxe9s-de-chiffrement-symuxe9trique}}

Un mécanisme similaire est utilisé pour le partage de clé de chiffrement
symétrique. Avant leur envoi vers le serveur, elles sont chiffrées avec
la clé publique de l’utilisateur destinataire. Ainsi seul l’utilisateur
destinataire peut déchiffrer la clé de chiffrement symétrique.

\hypertarget{cryptographie-symuxe9trique}{%
\subsection{Cryptographie
symétrique}\label{cryptographie-symuxe9trique}}

Nous avons choisi d’utiliser AES-GCM, un algorithme de chiffrement
symétrique qui permet de chiffrer et d’authentifier les données. Tout
comme pour la cryptographie asymétrique, nous avons utilisé la librairie
standard de JavaScript, \mintinline[]{c}{crypto}, pour chiffrer et
déchiffrer les données.

\hypertarget{hmac}{%
\section{Hmac}\label{hmac}}

Pour assurer que les logs n’aient pas été modifiés, nous avons utilisé
un HMAC. Celui-ci est généré à partir du contenu du log et d’une clé
secrète stockée sur le serveur. Le HMAC est stocké dans le log. Si le
message est modifié, le HMAC ne correspondra plus au contenu du log et
nous pourrons détecter qu’il y a eu une modification.

\hypertarget{fonctionnalituxe9s-de-lapplication}{%
\section{Fonctionnalités de
l’application}\label{fonctionnalituxe9s-de-lapplication}}

\hypertarget{authentification}{%
\subsection{Authentification}\label{authentification}}

\begin{figure}
\centering
\includegraphics{/tmp/tex2pdf.-9228386339f5691c/4eaf66a6cac0b369a5b3f5017e3dbb240406fbc5.png}
\caption{Authentification}
\end{figure}

Comme le montre le diagramme ci-dessus, lorsqu’un utilisateur souhaite
s’authentifier, il va envoyer au serveur la clé publique du device
depuis lequel il essaye de se connecter. Lorsque le serveur reçoit cette
clé publique, il vérifie d’abord si elle existe et va alors renvoyer à
l’utilisateur un challenge si cette clé publique existe bien dans la
base de données. Ce challenge est alors uniquement déchiffrable avec la
clé privée correspondante. Une fois le challenge déchiffré au serveur,
le serveur vérifie bien si le challenge est correct ou non. Si le
challenge est correct, alors l’utilisateur peut se connecter et reçoit
un JWT (Json Web Token). Dans le cas contraire, l’utilisateur ne peut se
connecter.

\newpage

\hypertarget{management-des-devices}{%
\subsection{Management des devices}\label{management-des-devices}}

L’application permet à chaque utilisateur d’ajouter autant de devices
qu’il le souhaite. Quand un utilisateur ajoute un device, une paire de
clés publique-privée est générée pour ce device. Cela se produit tout
d’abord lorsque l’utilisateur va créer son compte pour la première fois,
une paire de clés publique-privée est alors créé pour ce device.
Ensuite, une fois connecté sur son device, l’utilisateur va avoir la
possibilité d’ajouter d’autres devices. Une fois la paire de clés
générée, l’utilisateur va devoir valider le device qu’il vient d’ajouter
via un device de confiance. C’est-à-dire un device déjà “Trust” (de
confiance) par l’utilisateur.

Par exemple, si un utilisateur souhaite ajouter un 2e device, il va
devoir, sur son 1er device, cliquer sur le bouton “Trust” afin
d’affirmer qu’il a confiance en ce device. Une fois cela fait, il va
pouvoir se connecter à ce 2e device.

Une fois qu’un device est “Trust”, il faut alors que l’utilisateur
puisse avoir accès à ses photos sur ce nouveau device. Pour cela, chaque
clé symétrique pour chaque photo va être chiffrée avec la clé publique
de ce nouveau device. Pour les noms d’album, ils seront directement
chiffrés avec la clé publique du device.

Dans le cas où un utilisateur décide de ne plus “Trust” un device car il
l’a perdu où il n’y a plus accès, alors on ne peut ni avoir accès à ce
device, ni aux photos et albums accessibles précédemment via ce device.

Lors de chaque requête vers le serveur, on vérifie d’abord si le device
en question est “Trust” (de confiance). Dans notre code, chaque
procédure est une “protectedProcedure” qui s’assure que le device qui
est en train d’effectuer une requête est “Trust”. Il existe des
exceptions concernant le login et la connexion, sinon il serait
impossible d’accéder à l’application.

\newpage

\hypertarget{upload-de-photos-et-cruxe9ation-dun-album}{%
\subsection{Upload de photos et création d’un
album}\label{upload-de-photos-et-cruxe9ation-dun-album}}

\hypertarget{upload-de-photos}{%
\subsubsection{Upload de photos}\label{upload-de-photos}}

Tout d’abord, il est possible d’upload une photo sans que cette dernière
fasse partie d’un album. Lorsqu’une photo est upload par un utilisateur,
une entrée est ajoutée pour la table “Picture” et “SharedPicture”. Même
si la photo n’est pas encore partagée avec quelqu’un, une entrée existe
dans “SharedPicture”, qui est la table permettant de voir qui a accès à
la photo. Cette table va alors permettre de récupérer les photos de
l’utilisateur qui ne sont pas présentes dans un album.

Lorsqu’une photo est upload, une clé symétrique est générée. Cette clé
symétrique va alors être utilisée pour chiffrer la photo. Il n’existe
qu’une seule clé symétrique par photo même si la photo se trouve dans
différents albums. Il n’y a donc pas de clé symétrique par album, mais
bien par photo. Le fait d’avoir une clé symétrique par photo permet, en
cas de réussite d’un attaquant à récupérer cette clé, d’avoir accès
uniquement à la photo en question et non à l’entièreté d’un album.

Il existe un cas particulier, lorsqu’une photo est ajoutée à un album
avec une certaine personne et que cette dernière est déjà présente dans
un autre album avec la même photo qui est, du coup, déjà partagée avec
cette personne. Prenons 3 personnes : A, B et C. A et B ont un album en
commun avec une photo x ajoutée par A. Si A, B et C crée un album
commun, et que A décide d’y ajouter cette photo x, alors l’accès à la
photo n’est donné qu’à C étant donné que B et A possèdent déjà l’accès à
cette photo.

\newpage

\hypertarget{cruxe9ation-dun-album}{%
\subsubsection{Création d’un album}\label{cruxe9ation-dun-album}}

Comme expliqué précédemment, la clé symétrique permettant de chiffrer
les photos n’est pas basée sur un album, mais bien par photo. Lorsqu’un
utilisateur crée un album, l’application va alors faire une requête au
serveur afin de récupérer la clé publique de l’utilisateur afin de
chiffrer le nom de l’album. Le nom de l’album est quant à lui, chiffré
avec la clé publique du device de l’utilisateur. L’application envoie
alors au serveur le nom de l’album chiffré ainsi que l’id du device sur
lequel l’album a été créé. Le nom de l’album est alors déchiffré via la
clé privée qui est stockée localement sur le device de l’utilisateur. Il
s’agit du scénario classique lorsqu’un utilisateur possède un seul
device.

Le choix d’utiliser la clé publique du device de l’utilisateur pour
chiffrer le nom de l’album plutôt qu’une clé symétrique est que le
chiffrement du nom reste quelque chose d’assez rapide contrairement au
chiffrement des photos qui requière le chiffrement symétrique afin
d’être performant.

Dans la réalité, quand on un utilisateur crée un album, l’application va
faire une requête au serveur afin d’avoir toutes les clés publiques pour
chaque device que possède l’utilisateur. Avec ces clés publiques, le nom
de l’album est chiffré autant de fois qu’il y a de clés publiques. Un
dictionnaire est alors renvoyé au serveur. Ce dictionnaire est composé
d’un device id et d’un chiffrement lié à ce device via la clé publique
de ce dernier.

Exemple : deviceID1 = chiffrementPublicKeyDeviceID1 deviceID2 =
chiffrementPublicKeyDeviceID2 …

Selon le device utilisé par l’utilisateur, le nom de l’album est alors
déchiffré via la clé privée stockée localement.

\hypertarget{stockage-avec-minio}{%
\subsubsection{Stockage avec Minio}\label{stockage-avec-minio}}

Pour envoyer la photo chiffrée au serveur, il y a d’abord besoin de la
formater. Étant donné qu’en HTTP il n’est pas possible d’envoyer la
photo chiffrée sous forme d’un array buffer, il faut formater la photo
chiffrée en hexadécimal. Une fois les données relatives à la photo
stockées dans la base de données, la photo est stockée dans un serveur à
part (le serveur de fichier Minio).

Concernant l’intégrité des photos, au moment d’upload le chiffrement de
la photo, une transaction SQL est crée. Cette transaction va alors
permettre de s’assurer que l’upload de la photo sur le serveur de
fichier s’est bien déroulé. Avant de terminer la transaction, on
vérifie, via une méthode de Minio, si l’upload s’est bien déroulé et que
les données relatives à la photo sont bien ajoutées à la base de
données. S’il n’y a pas d’erreur, alors la transaction est validée et le
chiffrement de la photo se trouve bien sur le serveur de fichiers Minio.
En cas de problème, la transaction est annulée et les données relatives
à la photo ne sont ni ajoutées à la base de données ni au serveur de
fichiers. Cela permet de s’assurer que la photo soit bien ajoutée au
serveur de fichier de manière intègre.

\hypertarget{partage-des-photos-et-albums}{%
\subsection{Partage des photos et
albums}\label{partage-des-photos-et-albums}}

\hypertarget{partage-dune-photo}{%
\subsubsection{Partage d’une photo}\label{partage-dune-photo}}

Concernant le partage d’une photo, il faut d’abord récupérer les clés
publiques de l’utilisateur avec qui on souhaite partager la photo. Pour
cela, il y a une requête faite au serveur afin de récupérer toutes les
clés publiques de chaque device de l’utilisateur à qui on souhaite
partager notre photo. Le serveur va alors nous répondre avec un
dictionnaire contenant chaque id de chaque device avec sa clé publique
correspondante. Une fois les clés publiques récupérées, la clé
symétrique utilisée pour chiffrer la photo va alors être chiffrée autant
de fois qu’il existe de clé publique. Il y a alors un dictionnaire
renvoyé au serveur contenant l’id de chaque device avec le chiffrement
de la clé symétrique par device.

\begin{minted}[linenos,style=rainbow_dash,frame=lines,framesep=2pt,bgcolor=solbg,breaklines=true,autogobble]{c}
deviceID1 = chiffrementKeySymétricDeviceID1
deviceID2 = chiffrementKeySymétricDeviceID2
\end{minted}

De plus, il est tout à fait possible de partager une photo qui n’est
présente dans aucun album.

\begin{figure}
\centering
\includegraphics[width=0.7\textwidth,height=\textheight]{/tmp/tex2pdf.-9228386339f5691c/63818f26aded7a6d4b1489d21660635b14570bbf.png}
\caption{Partage de photo}
\end{figure}

Comme le montre le diagramme ci-dessus, l’utilisateur avec qui la photo
est partagée peut tout simplement déchiffrer la clé symétrique chiffrée
avec sa clé privée stockée localement afin d’avoir accès à la valeur de
la clé symétrique. Une fois cette valeur récupérée, il peut alors
déchiffrer la photo et y avoir accès.

\hypertarget{partage-dun-album}{%
\subsubsection{Partage d’un album}\label{partage-dun-album}}

Concernant le partage d’un album, il s’agit du même mécanisme que celui
décrit précédemment concernant le partage d’une photo. Il y a cependant
le nom de l’album qui lui aussi est chiffré via la clé publique du
device de l’utilisateur. Pour cela, il y a d’abord une requête au
serveur pour récupérer les clés publiques de l’utilisateur afin de
chiffrer le nom de l’album et ensuite une requête afin de récupérer les
clés publiques de l’utilisateur avec qui on souhaite partager l’album.
Comme expliqué précédemment, chacune des clés symétriques va alors être
chiffrée par les clés publiques existantes.

Comme expliqué précédemment, l’utilisateur avec qui l’album est partagé
va pouvoir déchiffrer le nom de l’album via sa clé privée.

\hypertarget{rate-limiting}{%
\section{Rate limiting}\label{rate-limiting}}

Comme cité plus haut dans le rapport, nous avons utilisé un système de
rate limiting pour éviter les attaques de type brute force et DOS(Denial
of Service). Son implémentation est assez simple, aucun package n’a été
utilisé car aucun ne correspondait à nos besoins. Nous avons donc
implémenté notre propre système de rate limiting. Celui-ci est un
middleware qui est appelé avant les requêtes. Il incrémente un compteur
stocké dans la base de données Valkey pour chaque adresse IP et chaque
endpoint. Si le compteur dépasse une certaine valeur, une erreur est
renvoyée. Voici le code de ce middleware :

\begin{minted}[linenos,style=rainbow_dash,frame=lines,framesep=2pt,bgcolor=solbg,breaklines=true,autogobble]{ts}
export const rateLimitedMiddleware = t.middleware(
  async ({ path, ctx, next }) => {
    const res = await ctx.cache.incr(`${path}:${ctx.ip}`);
    if (res === 1) {
      await ctx.cache.expire(`${path}:${ctx.ip}`, env.RATE_LIMIT_WINDOW);
    }
    if (res > env.RATE_LIMIT_MAX) {
      logger.error(`Rate limit exceeded for ${ctx.ip} on ${path}`);
      throw new TRPCError({ code: "TOO_MANY_REQUESTS" });
    }
    return next();
  },
);
\end{minted}

\hypertarget{protection-contre-les-injections-sql}{%
\section{Protection contre les injections
SQL}\label{protection-contre-les-injections-sql}}

Pour nous protéger contre les injections SQL, nous avons utilisé un ORM
(Object-Relational Mapping) nommé Prisma. Prisma est un ORM qui permet
de manipuler la base de données sans écrire de requêtes SQL. Celui-ci
est sécurisé par défaut et empêche les injections SQL.

\hypertarget{protection-contre-les-attaques-xss}{%
\section{Protection contre les attaques
XSS}\label{protection-contre-les-attaques-xss}}

Pour nous protéger contre les attaques XSS, nous utilisons la librairie
react-dom qui permet de manipuler le DOM de manière sécurisée. React DOM
échappe automatiquement les caractères spéciaux et empêche les attaques
XSS.

\hypertarget{analyse-de-code-snyk}{%
\section{Analyse de code Snyk}\label{analyse-de-code-snyk}}

Pour analyser notre code et détecter les vulnérabilités, nous avons
utilisé Snyk. Snyk est un outil qui permet de détecter les
vulnérabilités dans les dépendances de notre projet. Il scanne les
dépendances et les compare à une base de données de vulnérabilités. Si
une vulnérabilité est détectée, Snyk nous avertit et nous propose des
solutions pour la corriger. De plus, Snyk peut également être utilisé
pour détecter les vulnérabilités dans le code source.

\hypertarget{signature-des-commits-git}{%
\section{Signature des commits git}\label{signature-des-commits-git}}

Pour éviter une attaque de type “supply chain attack” ou d’usurpation
d’identité, nous avons mis en place une signature des commits git. Cela
permet de vérifier l’identité de l’auteur via sa clé GPG. Ainsi, si un
attaquant modifie le code source et ne signe pas le commit, il sera
facilement détecté.



\end{document}
